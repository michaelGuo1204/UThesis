% !Mode:: "TeX:UTF-8" 

\BiAppendixChapter{致\quad 谢}{Acknowledgements}

完成了论文的撰写,本毕业设计课题也算是取得了阶段性的胜利。在这个课题的研究过程中,我逐渐学会了如何利用现有资源独立解决一个实际问题。
然而,从一开始确定研究方向过程中的迷茫,再到研究过程中出现的曲折,到最后论文撰写阶段的不知所措,匮乏的研究经验也使我在这个课题上吃
了不少苦头。因此我要首先感谢在我毕业设计阶段给予我莫大帮助的郭燕老师以及课题组内的陈佳斌、刘聪聪、姚石师兄以及陈一宵师姐。感谢你们
在完成这个毕设课题的过程中给予我的灵感与帮助。

敲下毕业论文的最后一个字之后,纵使有千般不舍,四年的大学生活还是走向了终点。四年光阴中我有过迷惘、有过感伤,但是更有快乐、有希望。
感谢为我传道授业解惑的郭燕、杨铁林老师,感谢您在专业领域对我的指点与教导,让我最终坚定地走上了生物信息学的道路。感谢四年大学生涯中
带领我博览生物学与化学奥秘的卢晓云、谭丹、和玲、吴超与邵永平等老师,你们精彩的课堂与严谨治学的态度让我收获颇丰。同时也感谢陈昱杰、王炜喆、刘竺航、杨明泽、张宇博等东12大家庭中的好朋友的陪伴,
在繁重课业压力之下,是你们带给了我无尽的快乐。感谢化生81班大家庭中的全体成员对我的支持与对化生班委会工作的认可,特别感谢高旭帆同学,与他共事
时见证的他严谨认真的学习与科研态度将会是我一生的榜样。感谢何千越、陈磊、邓晓东、罗一鸣、李泊宇与袁震宇等和我一同参加Robomaster比赛的队友与伙伴,虽然参赛过程中也有波折
,但我从他们身上所学到的知识、所感受到的对技术的热爱、所经历的备赛时光都将会是我大学期间一段美好的回忆。感谢上官蒙蒙与陈龙辅导员对我生活的照顾,辛苦了!
最后还要感谢一直陪伴着我的挚友丁晓宇和李硕,与你们的友谊是我的幸运,祝你们前程似锦!

特别要感谢我的所爱吕瀛玥同学,她让我的努力与坚守都有了意义。高考时命运和我们开了一个天大的玩笑,让我们到相距千里的两座城市开始自己的大学生活。
但正如身无彩凤双飞翼,心有灵犀一点通,我们的付出最终让希望的曙光驱散了所爱隔山海的阴霾。感谢你在我们一起相处的时光中对我的包容与付出,我们已经度过的五年青春以及
未来的厮守与陪伴将会是我永久的力量源泉!

感谢我的爷爷奶奶、爸爸妈妈。蓼蓼者莪,匪莪伊蒿;哀哀祖亲,育我劬劳!我今天的一切离不开爷爷奶奶从小对我的教导与鞭策,他们不仅给了我一个快乐的童年,还克服万难拖着日趋老迈的身体陪我到异乡求学。
也要感谢爸爸妈妈生我养我之恩、育我教我之劳。你们的付出我看在眼里,记在心上。感谢外婆与已经永远离开我的外公,你们给我的童年增添了无数的快乐,
也在我人生路上的重大选择之前给了我受益终身的建议。也感谢其他给予我帮助过的家人,谢谢你们!

寥寥数语说不完情深义厚;短短几行道不尽恩重如山。临别之际,祝大家平安喜乐、幸福安康!

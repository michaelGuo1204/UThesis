\chapter{总结与展望}

\section{工作总结}

构建高效可解释骨关节炎对骨关节炎的早期诊断与防治有着重要的意义。本文首先根据目前已发表的GWAS研究及公共数据库UK BioBank获取患者表型与基因型数据并进行数据预处理;之后构建了包括图结构估计器、图卷积神经网络、表型信息融合与图解释器四个模块在内的骨关节炎风险预测模型;最后本文对该模型的性能以及预测结果进行了进一步的分析和解读,继而对模型的预测准确性,可解释性等指标进行评估。本文对主要完成的工作总结如下。

1. 构建了适用于图分类问题的图结构估计器:目前提出的用来处理无结构数据的图结构估计器多适用于图节点分类问题,对适用于图分类问题的图结构估计器研究较少。本文提出了一种基于变分期望最大化算法的图结构估计器,该估计器通过对图神经网络输出的处理通过参数估计的方法预测图节点之间关联可能性与图整体结构,继而在图神经网络训练中动态更新图结构。经测试,该估计器能明显改善图分类问题中无结构数据在图神经网络上的表现。同时,参数估计方法使得估计器生成的图结构具有明显的局部结构,能够用于图深层信息的挖掘。该估计器弥补了图分类问题图估计器的空白,对无结构数据的图神经网络处理具有积极意义。

2. 构建了具有实用价值的骨关节炎风险预测模型:现有的骨关节炎风险预测模型效果差,不能满足骨关节炎患者早期诊断与筛查的需求。本文所构建的风险预测模型通过结合表型信息与基因型信息输入图神经网络进行处理,最终实现了较好的预测准确率,具备实用风险预测价值。同时本文提出模型通过结合图解释器对风险预测结果结合图神经网络给出预测值解释,使得模型在输出预测患病风险的同时能够对潜在的病因进行预测,对疾病分型与精准治疗具有积极意义。

3. 通过估计器所得图结构界定出骨关节炎两可能诱因:通过对图估计器所得图结构并结合PheWAS表型信息研究,我们发现了图结构中的两典型簇。其中一簇主要与受教育程度、是否从事中体力劳动等环境因素相关,本文推测该簇内SNP可能与环境因素导致关节磨损继而导致的骨关节炎相关;另一簇主要与免疫、内分泌等自身因素相关,本文推测该簇主要与个体自身肥胖导致的糖尿病与超重继而诱发的自身型骨关节炎相关。本文因此认为骨关节炎存在包括重体力劳作在内的环境诱因与包括肥胖、二型糖尿病在内的自身诱因,与目前对骨关节炎研究的观点一致。以上分析均是在单纯依靠图结构的基础上进行的,可以推广到其他疾病风险预测模型的研究之中,对复杂多基因疾病的风险预测与分型诊断有着积极意义。

\section{展望}

虽然本文所提出风险预测模型在给出可信预测结果的同时兼具可解释性,但是本文中还存在若干问题需要进一步研究

1. 模型超参数与结构还存在调整空间:本模型中四个模块都有着数量较多的决定模型性能的超参数,由于时间限制本文未能对这些超参数进行细致调整,可能对最终模型的性能产生不利影响。后续研究中仍要对其中一些参数加以分析调整以求继续提高模型性能。同时随着图神经网络研究的快速发展,本研究进行之时不断有效果更好的图卷积方式出现,后续研究中还需基于这些研究对图卷积层进行优化。

2. 图估计器仍需继续研究:为了简化问题,本模型图估计器所生成的图结构中各边的权重一致。但是现实生活中的图数据中各边权重通常不一致,并且有着具体意义,忽略该权重意味着潜藏信息的损失。如何通过参数描述各边的权重并通过算法估计该参数仍需后续研究。

3. 图估计产生图结构仍需进一步分析:本研究中图估计器生成的图结构共分为六簇,但是由于SNP位点生物学意义的复杂性,本文只对其中两簇进行了浅显的分析,并且没有对簇内节点之间与簇与簇之间的关联加以深入解释。后续研究中仍要对产生该结构的具体原因与意义加以具体分析

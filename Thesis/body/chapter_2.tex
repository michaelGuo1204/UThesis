% !Mode:: "TeX:UTF-8" 

\BiChapter{论文结构}{Structure}

\BiSection{摘要}{Abstract}

摘要是论文的高度概括,是全文的缩影,是长篇论文不可缺少的组成部分。要求用中、英文分别书写,一篇摘要不少于 400 字。英文摘要与中文摘要的内容和格式必须一致。

\BiSection{主要符号表}{Symbol Table}

如果论文中使用了大量的物理量符号、标志、缩略词、专门计量单位、自定义名词和术语等,应将全文中常用的这些符号及意义列出\footnote{\url{https://es.overleaf.com/learn/latex/List_of_Greek_letters_and_math_symbols}}。 \textbf{如果上述符号和缩略词使用次数不多,可以不设专门的主要符号表,但在论文中出现时须加以说明。}论文中主要符号全部采用法定单位,严格执行国家标准(GB3100~3102—93)有关“量和单位”
的规定。单位名称采用国际通用符号或中文名称,但全文应统一,不得两种混用。 缩略词应列出中英文全称。

\BiSection{正文}{Main body}

\BiSubsection{绪论}{Introduction}

绪论相当于论文的开头,它是三段式论文的第一段(后二段是本论和结论)。绪论
与摘要写法不完全相同,摘要要写得高度概括、简略,绪论可以稍加具体一些,文字以 1000 字左 右为宜。绪论一般应包括以下几个内容: 

(1)为什么要写这篇论文,要解决什么问题,主要观点是什么。

(2)对本论文研究主题范围内已有文献的评述(包括与课题相关的历史的回顾,资料来源、性质
及运用情况等)。

(3)说明本论文所要解决的问题,所采用的研究手段、方式、方法。明确研究工作的界限和规模。
概括论文的主要工作内容。

\BiSubsection{课题的研究方法与手段}{Methodology}
课题研究的方法与手段,分别以下面几种方法说明: 用实验方法研究课题,应具体说明实验用的装置、仪器、原材料的性能等是否标准,并应对所
有装置、仪器、原材料做出检验和标定。对实验的过程和操作方法,力求叙述得简明扼要,对实验 结果的记录、分析,对人所共知的或细节性的内容不必过分详述。 用理论推导的手段和方法达到研究目的的,这方面内容要精心组织,做到概念准确,判断推理
符合客观事物的发展规律,要做到言之有序,言之有理,以论点为中心,组织成完整而严谨的内容 整体。
用调查研究的方法达到研究目的的,调查目标、对象、范围、时间、地点、调查的过程和方法
等,这些内容与研究的最终结果有关系,但不是结果本身,所以一定要简述。但对调查所提的样本、 数据、新的发现等则应详细说明,这是结论产生的依据。

\BiSubsection{结论与展望}{Conclusion}

在写作时,应对研究成果精心筛选,把那些必要而充分的数据、现象、样品、认识等选出来,
写进去,作为分析的依据,应尽量避免事无巨细,把所得结果和盘托出。在对结果做定性和定量分 析时,应说明数据的处理方法以及误差分析,说明现象出现的条件及其可证性,交代理论推导中认 识的由来和发展,以便别人以此为根据进行核实验证。对结果进行分析后所得到的结论和推论,也 应说明其适用的条件和范围。恰当运用表和图作结果与分析,是科技论文通用的一种表达方式。

结论与展望:结论包括对整个研究工作进行归纳和综合而得出的总结;所得结果与已有结果的
比较;联系实际结果,指出它的学术意义或应用价值和在实际中推广应用的可能性;在本课题研究 中尚存在的问题,对进一步开展研究的见解与建议。结论集中反映作者的研究成果,表达作者对所 研究课题的见解和主张,是全文的思想精髓,是全文的思想体现,一般应写得概括、篇幅较短。

\BiSubsection{致谢}{Thanks}

对于毕业设计(论文)的指导教师,对毕业设计(论文)提过有益的建议或给予 过帮助的同学、同事与集体,都应在论文的结尾部分书面致谢,言辞应恳切、实事求是。


\BiSection{参考文献与注释}{Reference}

参考文献是为撰写论文而引用的有关文献的信息资源。
参考文献列示的内容务必实事求是。论文中引用过的文献必须著录,未引用的文献不得虚列。
遵循学术道德规范,杜绝抄袭、剽窃等学术不端行为。
参考文献须是作者亲自考察过的对学位论文有参考价值的文献。
参考文献应有权威性,应注意所引文献的时效性。 参考文献的数量:一般不少于 10 篇,其中,期刊文献不少于 8 篇,国外文献不少于 2 篇,均以近 5 年的文献为主。 注释是正文需要的解释性、说明性、补充性的材料、意见和观点等。


参考文献格式应符合国家标准 GB/T-7714-2005《文后参考文献著录规则》。中国国家标准化管理委员会于2015年5月15日发布了新的标准 GB/T 7714-2015《信息与文献参考文献著录规则》。因为二者的差别非常小,所以采用了新的标准。标准的 BiBTeX 格式网上资源非常多,本文使用了李泽平开发的版本~\upcite{Lee2016}。本模板中使用 bib\LaTeX  提供参考文献支持。请使用\texttt{\textbackslash upcite} 命令进行引用。

如,本文参考了如下文献\upcite{zhou2020egoplanner} \upcite{thrun2002probabilistic}。



\BiSection{装订要求}{Bindings}

论文的查重部分和不查重部分的内容如表\ref{tab_check}所示。

\begin{table}[!h]
	\renewcommand{\arraystretch}{1.2}
	\centering\wuhao
	\caption{论文查重部分与不查重部分内容对照表} \label{tab_check} \vspace{2mm}
	\begin{tabularx}{\textwidth} { >{\centering\arraybackslash}X >{\centering\arraybackslash}X }
	\toprule[1.5pt]
		查重部分 & 不查重部分 \\
	\midrule[1pt]
		封面        &   致谢                \\
		中文摘要    &   外文原文(附录)    \\
		英文摘要    &   外文译文(附录)    \\
		目录        &   有关算法(附录)    \\
		正文        &   有关图纸(附录)    \\
		参考文献    &   计算机源程序(附录)\\    
	\bottomrule[1.5pt]
	\end{tabularx}
\end{table}

\noindent \textbf{毕业设计(论文)装订次序要求} \\
第一 ~ 封面 \\
第二 ~ 任务书(双面打印) \\
第三 ~ 考核评议书(背面是答辩结果) \\
第四 ~ 中文摘要 \\
第五 ~ 英文摘要 \\
第六 ~ 目录 \\
第七 ~ 正文(含绪论和结论) \\
第八 ~ 致谢 \\
第九 ~ 参考文献 \\
第十 ~ 附录(含外文原文及其译文、有关图纸、计算机源程序等)
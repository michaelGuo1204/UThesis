% !Mode:: "TeX:UTF-8"

\usepackage[a4paper,top=21.5mm,bottom=19.5mm,left=26mm,right=26mm,includehead,includefoot]{geometry}					% 控制页面尺寸
\usepackage{titletoc}       % 控制目录的宏包
\usepackage{titlesec}       % 控制标题的宏包
\usepackage{fancyhdr}       % 页眉和页脚的相关定义
\usepackage{color}          % 支持彩色
\usepackage{graphicx}		% 处理图片
\usepackage{tabularx}		% 可伸缩表格
\usepackage{multirow}       % 表格可以合并多个row
\usepackage{booktabs}       % 表格横的粗线;\specialrule{1pt}{0pt}{0pt}
\usepackage{longtable}      % 支持跨页的表格。
\usepackage{enumitem}       % 使用enumitem宏包,改变列表项的格式
\usepackage{amsmath}        % 公式宏包
\usepackage{amssymb}		% 符号宏包
\usepackage{bm}				% 处理数学公式中的黑斜体的宏包
\usepackage{lmodern}		% 数学公式字体
\usepackage[amsmath,thmmarks,hyperref]{ntheorem}	% 定理类环境宏包
\usepackage[hang]{subfigure}						% 图形或表格并排排列
\usepackage[subfigure]{ccaption}					% 支持双语标题
\usepackage[sort&compress,numbers]{natbib}			% 支持引用缩写的宏包
\usepackage{fontspec}		% 字体设置宏包
%\usepackage{etoolbox}		% primarily towards LaTeX class and package authors.
%\usepackage{xltxtra}		% 提供了针对XeTeX的改进,自动调用xunicode宏包

% 生成有书签的 pdf 及其开关, 该宏包应放在所有宏包的最后, 宏包之间有冲突
\usepackage[xetex,
            bookmarksnumbered=true,
            bookmarksopen=true,
            colorlinks=false,
            pdfborder={0 0 1},
            citecolor=blue,
            linkcolor=red,
            anchorcolor=green,
            urlcolor=blue,
            breaklinks=true
            ]{hyperref}

% 算法的宏包,注意宏包兼容性,先后顺序为float、hyperref、algorithm(2e),否则无法生成算法列表
\usepackage[plainruled,linesnumbered,algochapter]{algorithm2e}

\usepackage{listings}		% 为了避免与页眉的兼容问题可将listings放入table环境中
\definecolor{codegreen}{rgb}{0,0.6,0}
\definecolor{codepurple}{rgb}{0.58,0,0.82}
\lstset{%
	language={[ISO]C++},				% 设置语言
	alsolanguage=Matlab,				% 可以添加多个语言选项
	alsolanguage=Verilog,
	morekeywords={numerictype,fimath,fipref,fi,trh},
	commentstyle=\color{codegreen},		% 注释颜色
	keywordstyle=\color{blue},			% 代码关键字颜色
	stringstyle=\color{codepurple},		% 代码中字符串颜色
	frame=single,			% 设置边框, 
	xleftmargin=1.7em,		% 设定listing左边空白 
	numbers=left,			% 左侧显示行号
	numberstyle=\small,		% 行号字体用小号
	breaklines=true,		% 对过长的代码自动换行 
	columns=flexible,		% 调整字符之间的距离
	tabsize=4				% table 长度
}

% 带圆圈的脚注
\usepackage{pifont}
\usepackage[flushmargin,para,symbol*]{footmisc}
\DefineFNsymbols{circled}{{\ding{192}}{\ding{193}}{\ding{194}}
	{\ding{195}}{\ding{196}}{\ding{197}}{\ding{198}}{\ding{199}}{\ding{200}}{\ding{201}}}
\setfnsymbol{circled}

% 字体设置,Windows自带黑体(simhei.ttf)和宋体(simsun.ttf)
\defaultfontfeatures{Mapping=tex-text}
\setmainfont{Times New Roman}
\setsansfont{Arial}
\setCJKmainfont{SimSun.ttf}
\setCJKsansfont{SimHei.ttf}
\setCJKfamilyfont{hei}{SimHei.ttf}
\newcommand{\hei}{\CJKfamily{hei}}

% 外文文献pdf插入
\usepackage{pdfpages}

% 符号表